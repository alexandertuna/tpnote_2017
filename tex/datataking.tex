\section{Data taking}
\label{sec:data-taking}

The data described in this note were collected from May 2017 until July 2017. Each continuous period of data-taking is
 referred to as a ``run'', and the runs are listed in Table~\ref{tab:runs}. 
 One notes that here are many more MMTP events than MMFE8 events. There are several  reasons
 for this. The MMTP sends data whenever the finder finds a trigger, whereas the MMFE8 only sends
 data only when it receives a trigger signal from the scintillator (the scintillator trigger rate is about 1 Hz rate,
 and 60\% of the triggers contain a good track with 4 or more clusters and triggerable).

 For a good track, the MMTP  typically produces a handful of triggers 
 due to the overlapping roads. For example, a downward-going muon passing entirely through road 2
 also generates triggers in roads 1 and 3. In addition, if a track has more hits than the minimum required by the finder,
 when some of these hits get old additional triggers can be produced.
 A large number of triggers can be also produce by non-random noise.
 This seem to be the case for  run 3522 and 3530.
 This started a long investigation of the noise source. The cause has been now identified as
 EMI produced by the  the ART clock transmitted from the MMFE8 to the ADDC through an unshielded pairs of wires
 in the miniSAS cable and by the AC/DC converter powering the ADDC. 

 Between Runs 3522 and 3527, two front-end cards were repaired and replaced on the detector. Therefore, only
 data from Runs 3527, 3528, and 3530 are used to compare
the MMTP performance using different peaktimes in Sec.~\ref{sec:perf-integ}
%%%%%%%%%%%%%%%%%%%%%%%%%%%%%%%%
\begin{table}[!htpb]
\begin{center}
  \begin{tabular}{c | c | c | c | c}
    Run  & Dates         & MMFE8 Events & MMTP Events & Notes \\
    \hline
    3522 & 05/11 - 05/18 & 296532       & 15774567    & 200 ns integration time \\
    3527 & 06/26 - 07/02 & 233645       & 3635093     & 100 ns integration time \\
    3528 & 07/02 - 07/07 & 205053       & 3253330     & 50 ns integration time \\
    3530 & 07/19 - 07/24 & 192114       & 4060891     & 200 ns integration time \\
  \end{tabular}
  \caption{Run information.  }
\label{tab:runs}
\end{center}
\end{table}
%%%%%%%%%%%%%%%%%%%%%%%%%%%%%%%%%%%%%%%%%%%%%%%%%%%%

\subsection{Combining data streams}
\label{sec:data-streams}
As already mentioned, five data streams are produced and recorded on disk. In addition  to the scintillator and MMFE8 data streams,
the MMTP algorithm writes a data stream (TPFIT) with the trigger properties - including its BCID  as explained earlier,
 a data stream (GBT) with ADDC data in the 15 the BC preceding the
MMTP 
trigger, and a data stream (SCTBC) containing  the the scintillator trigger time converted into a 16-bit  BCID. 

Scintillator data, MMFE8 data, and  SCTBC data
 are easily combined (SC+MM data) using the event number and occasionally the DAQ time stamp.
Because the MMTP rate is much higher, we combine it with  SC+MM data through several steps: a) we select SC+MM data which contain MM clusters
 compatible with a track
and which also satisfy the MMTP finder; d) we select TPFIT events first with a DAQ time stamp within 100 ms from that of the SC+MM event and then
in a 3 BC window around SCTBC; c) of the remaining events in the TPFIT data, we pick the one which has more VMMs and boards in common with the MM clusters.

To associate GBT events to TPFIT events, we use the event number and BCID comparisons between BCID  trigger and ADDC hits.


