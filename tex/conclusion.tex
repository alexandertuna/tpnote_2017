\section{Conclusion}
\label{sec:conclusions}

The performance of the Micromegas Address in Real Time (ART) and trigger processor is shown. The performance is measured with hundreds of thousands of cosmic muons in a low-background environment at the Harvard cosmic ray test stand. The test stand employs a full trigger electronics path with prototype hardware: MMFE8s equipped with VMM2, the FPGA-based ADDC V1, and a Micromegas trigger processor (MMTP) implemented on a VC707 FPGA evaluation board.

Given a track identified by the full front-end MMFE8 readout, the efficiency of the Micromegas trigger is $>\! 99\%$. The spatial, angular, and time resolution of the MMTP is measured with the front-end readout as reference, and the spatial and angular resolutions are found to be comparable to predictions made in the TDR. The time resolution is worse than predicted. The time resolution improves with shorter integration time; however, even with the shortest integration time considered, the time resolution is worse than anticipated.

