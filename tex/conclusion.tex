\section{Conclusion}
\label{sec:conclusions}
 Using millions of cosmic tracks, we have commissioned a prototype of Micromegas trigger processor,
 and measured its performance using tracks reconstructed with Micromegas clusters.
 We use trigger roads as large as  1 VMM, and offline we reduce them to 1/4 VMM to evaluate
 the NSW performance.

 In this condition, 

 a) the $\theta_{\rm global}$ accuracy is found to be 0.06 mrad,  better than what is needed;

 b) the $\theta_{\rm local}$ accuracy is spoiled by $\delta$ rays. A 15 mrad cut on the difference of these two angles
 reduces the MMTP efficiency by 4\%;

 c) we did not measure directly the MMTP $\phi$ resolution. From the accuracy in measuring
 the non-precision coordinate, we estimate that the MMTP $\phi$ resolution  is 33 and 14 mrad for the shortest (which are closest to the beam line) and longest  Micromegas readout strips, respectively. 

 Considering that a NSW module will hardly reach a 100\% efficiency, it seems prudent
 to start accepting that the MMTP algorithm will need to use a 7-8 BC window, independent of the
 VMM  peaktime.

 In our measurements, the rate of uncorrelated noise is about 2  Hz per strip.
 Using a large BC window, the MMTP spatial accuracy will be further degraded by accidental hits at the highest LHC planned
 luminosity.   One expects a rate of
 about 40 kHz per strip, almost independent of the distance from the beamline.

 We are presently simulating the MMTP response in this high background environment.
 In particular, we are investigating the use of $\leq$1/8 VMM  roads for forming triggers.
 Such narrow roads might not require the use of local fits to reduce the background, thus eliminating the bottle-neck of 
the fitter (only a few found triggers can be fitted at each BC) and also  reducing the trigger latency.


