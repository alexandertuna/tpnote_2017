\documentclass{ATLAS_latex/atlasnote} 
\skipbeforetitle{-5pt}

\usepackage{graphicx,multirow}
\usepackage{epstopdf}
\usepackage{authblk}
\usepackage{hyperref}
\usepackage{pdfpages,subfigure,caption,placeins}
\usepackage{color}

\newcommand{\red}[1]{\textcolor[rgb]{1,0,0}{#1}}

\input{tex/definitions}
\input{tex/prd_definitions}

\renewcommand{\topfraction}{0.95}
\renewcommand{\bottomfraction}{0.8}
\renewcommand{\textfraction}{0.06}
\renewcommand{\floatpagefraction}{0.85}
\setcounter{totalnumber}{5}
\setcounter{topnumber}{4}

\bibliographystyle{apsrev}

\title {Performance of a Micromegas Trigger Processor with Cosmic Ray Muons}

\usepackage{authblk}
\renewcommand\Authands{, } % avoid ``. and'' for last author
\renewcommand\Affilfont{\itshape\small} % affiliation formatting
\author[a]{N.~Felt}
\author[a]{M.~Franklin}
\author[a]{P.~Giromini}
\author[a]{J.~Philion}
\author[a]{C.~Rogan}
\author[a]{A.~Tuna}
\author[a]{A.~Wang}
\affil[a]{Harvard University, Cambridge, Massachusetts 02138, USA}

\abstracttext{\noindent We built a strong, beautiful, and brave Micromegas octuplet, and we instrumented it with MMFE8 front-end boards, ART Data-Driver Cards, and a Trigger Processor. The trigger performance for $\ge 0.8$~GeV$/c^2$ cosmic muons is described.}

\clearpage

\begin{document} 

\setcounter{page}{2}

\section{Introduction}
\label{sec:intro}

We investigate the response of the micromegas Address in Real Time (ART) and trigger processor.



\section{Cosmic muon telescope and Micromegas octuplet}
\label{sec:exp}

The Micromegas and scintillator detectors of the Harvard cosmic ray test stand are described in detail in previous notes presenting 
the performance of the full MMFE8 readout~\cite{noisy,noiseless}. Eight layers of Micromegas detectors are stacked vertically for the
 purpose of providing eight measurements of the position of a cosmic muon as it travels through the experiment.
 The readout strips of each layer have a 0.4 mm pitch, and are arranged according to the NSW design: $XXUV-UVXX$, in which strips in the $X$ planes run
 perpendicular to the readout edge of the chamber, and strips in the $U$ ($V$) planes are tilted at
 an angle of $1.5^\circ$ ($-1.5^\circ$) to provide a measurement of the non-precision $y$ coordinate. Each Micromegas detector covers an active area of
 200 x 204.6 mm$^2$.
The type and $z$~position of the different readout
 planes are summarized in Table~\ref{tab:tab_1}.
%%%%%%%%%%%%%%%%%%%%%%%%%%% 
\begin{table}
 \begin{center}
% \begin{ruledtabular}
 \begin{tabular}{lccc}
 \hline \hline
  Readout plane  &  Type & $z_{\rm board} (mm)$ &  $z_{\rm bary} (mm)$      \\
  0      & $X$   &  0.0 &  -2.7     \\ 
  1      & $X$  & 11.2 &   13.9  \\
  2      & $U$   &32.4 &   29.7 \\
  3      & $V$   & 43.6 &    46.3   \\
  4      & $U$   & 113.6 &    110.9    \\
  5      & $V$   & 124.8 &   127.5  \\
  6      & $X$   & 146.0 &    143.3   \\
  7      & $X$   & 157.2 &    160.0   \\
 \hline \hline
 \end{tabular}
% \end{ruledtabular}
 \caption{ Characteristics of the 8 Micromegas planes assembled into 2 quadruplets (see text).
 Planes $x$ have electrodes parallel to the scintillator $x$ coordinates. Electrodes of planes $u$ and $v$ are 
tilted with respect to the $x$ electrodes by $\pm 1.5 \deg$, respectively. The  positions
along the $z$ axis of each electrode plane are called  $z_{\rm board}$. 
The $z$ position at half drift gap, where the barycenter is reconstructed, is called $z_{\rm bary}$. 
 }
\label{tab:tab_1}
 \end{center}
\end{table}
Three layers of scintillators counters -  placed above, on top, and below the octuplet -  provide a trigger
for the full MMFE8 readout. A scintillator trigger requires the coincidence of signals in all three layers. 
 Concrete blocks, 5 feet thick, are placed between the octuplet and the bottom scintillator to harden the muon sprectrum.

\subsection{Readout electronics}
\label{sec:exp-elx}
The Micromegas readout electronics consists of three cards. The MMFE8 card is a demonstrator that houses 8 VMM2 ASICs per card.
 These ICs provide digitized  arrival-time and charge information for the signal of each  Micromegas readout strip.  
 A description of the MMFE8 boards can be found in Refs.~\cite{noisy,mmfe8}. A study of its performance is in
 Refs.~\cite{noisy,noiseless}. The MMFE8 demonstrator uses a FPGA for configuring
 and reading out the VMMs digitized information  over ethernet.
 We use 8 MMFE8s to read the 512 channels of  the 8 layers of Micromegas in the octuplet.

 The ADDC card~\cite{nswtdr} receives data from 4 MMFE8s over miniSAS connections and packages this data
 into a single output for the MMTP board. For this purpose, the ADDC V1 uses a FPGA communicating with a GBTx chip.
 As mentioned earlier, we need two such boards.

The Micromegas Trigger Process (MMTP) is the last card of the Micromegas trigger path. It receives data from ADDCs over SFP+  optical connections and
 searches the data for the presence of hits consistent with a track segment using firmware running on a  VC707 FPGA evaluation board.
  Pictures of the ADDC V1 and the demonstrator MMTP are shown in Figure~\ref{fig:cards}.

\begin{figure}[!htpb]
  \begin{center}
    \includegraphics[width=0.48\textwidth]{figures/photos/IMG_0840.JPG}
    \includegraphics[width=0.48\textwidth]{figures/photos/IMG_0836.JPG}
  \end{center}
  \vspace{-10pt}
  \caption{Pictures of the ADDC V1  (left) and the MMTP board (right).}
  \label{fig:cards}
\end{figure}

An additional board, referred to as  clock/trigger card, has been designed  by J.~MacArthur.
 This board generates and distributes a 40 MHz clock to all MMFE8 and MMTP boards.
 Time is measured in units of BC, where BC is a tick of this clock.
 The content of counters incremented using this common external clock, referred to as BCID and
 located in all boards, is attached to the event information.
The clock/trigger cards also receive and distribute scintillator-trigger signals   
as discussed in Section~\ref{sec:exp-mmfe}. 
A  flow diagram of the hardware is shown in Figure~\ref{fig:cartoon_elx}.

\begin{figure}[!htpb]
  \begin{center}
    \includegraphics[width=1.0\textwidth]{figures/cartoons/electronics_path.pdf}
  \end{center}
  \vspace{-20pt}
  \caption{Schematic of the data flow in the readout electronics used at the Harvard cosmic ray test stand. 
    Inspiration for the cartoon was provided by L. Levinson.}
  \label{fig:cartoon_elx}
\end{figure}

\subsection{Trigger data path}
\label{sec:exp-art}

The MMTP data path starts with the ART signal provided by each VMM.
The VMM has two operating modes for the ART signal: the ART flag can be issued when the integrated charge of a signal is above threshold ($\simeq$ 3 fC) or reaches its peak value 
 ($\simeq$ chosen peaktime, usually 200 ns). We choose the first mode to reduce the latency of the Micromegas trigger which is already
 quite close to the allocated 1 $\mu$s~\cite{nswtdr}.

The ART flag is followed by the serialized address of the channel (6-bits) released at rising and falling edges of the
 MMFE8 160 MHz ART clock,  derived by boosting up  the external BC clock.
At the flag falling edge,
the ADDC uses the MMFE8 ART clock to deserialize the ART address. This clock, scaled down to 40 MHz, also increments a counter, the value of which
at the flag  is attached to the channel address together with  MMFE8 board and VMM numbers (ART string). 
 At each  BC, the ADDC uses  GBTx frames to collapse the ART informations of 32 VMMs into 112-bits packets which
are sent through a SFP+ optical link to the MMTP board. Each frame has only space for 8 ART strings. If more strings are present,
 the ADDC keeps the eight with highest priority, where the priority is assigned according to the VMM position. While this is not a problem in our case,
this step will need re-evaluation in the LHC high-L situation.

 The  e-link between each MMFE8 and the ADDC is provided by miniSAS 36-line cables in which each VMM ART output uses a preassigned pair of cables.
 Unfortunately, the ART clock uses pairs of  unshielded sideband lines which generate quite a bit of
 EMI captured by the octuplet Al frame and in turn injected into the Micromegas readout electrodes.
 It took us a few months to discover it and one day to fix it.

  



  The ADDC V1 were built by BNL and the firmware to receive the data, align  and format them was imported by Lin Yao from the ART ASIC
architecture\cite{addc}.
 Two mistakes were found and fixed in the firmware: a) an error in deserializing ART data with the 160 MHz clock which produced
 incorrect ART addresses; and b) the ART flag, variable in length, was not always detected resulting into a loss of events.
 Other than that, the ADDC firmware was found to be flawless.
 The MMTP process uses the ADDC information to find and characterize trigger candidates.   

 
As in the NSW design, the 2 ADDCs are placed near the octuplet and the VC707 board receiving
 their data is placed in a 10 m away shack -elevated to control room status- 
 which also houses the DAQ computer and scintillator trigger/readout electronics. 

\subsection{MMFE8 and scintillator data path}
\label{sec:exp-mmfe}
The ART data is distinct from the full readout of the MMFE8, as the latter includes detailed information about the charge and arrival
time of all signals captured during one event. This event information is used to evaluate the MMTP efficiency, and
time and spatial resolution independently of the octuplet efficiency.
The MMFE8 and scintillator data paths are described in detail in Refs.~\cite{noisy,noiseless}. Upon a scintillator trigger, the DAQ computer pulls out 
 the MMFE8 information recorded  in 100 $\mu$s before the trigger using an ethernet connection. 
 The arrival time of the scintillator signals are also recorded
 and sent over ethernet to the DAQ computer. Micromegas and scintillator data, as well MMTP data, are recorded event by event with the DAQ-computer time stamp.
 This allows us to recombine the information of a single event split into different files. 

 The scintillator trigger, which has a 1.2 ns resolution, is also sent to the MMTP where it is stamped with content of a counter incremented by  the  640 MHz 
clock derived from  the external BC clock.





\section{MM TP Algorithm}
\label{sec:alg}

The MM TP algorithm is designed to receive ART data transmitted via optical fiber using the GBT protocol, decode the ART data, filter the ART data into subsets (``triggers'') of ART data which satisfy a spatial and temporal coincidence, and calculate quantities which describe these triggers. The first step is referred to as the \textit{decoder}, the second step is referred to as the \textit{finder}, and the third step is referred to as the \textit{fitter}.

The MM TP algorithm is written in firmware and placed on a Xilinx Virtex-7 FPGA, which is housed by a VC707 evaluation board. The algorithm meets timing requirements for synthesis and implementation of the algorithm on the FPGA. Ultimately, the algorithm will be placed on a larger FPGA on a high-density optical mezzanine card (``HORX'') which will be housed in an ATCA shelf. The VC707 implementation is already capable of testing many aspects of the algorithm, however, including the steps mentioned above.

\subsection{Decoding}
\label{sec:alg-decode}

ART data is transmitted from the ADDC to the MM TP using the GBT protocol via optical fibers. The data is transmitted as a 128-bit word per BC. 12 bits describe the BCID; 32 bits describe which of the 32 VMMs associated with one ADDC are transmitting an ART signal; 48 bits give a maximum of 8 ART strip numbers, which are 6 bits each; and the remaining bits are used for checking data quality, such as an 8-bit data parity word.

After receiving the ART data, the MM TP converts the VMM and strip numbers into global strip numbers, also taking into consideration that some of the MM chambers are flipped relative to each other. For example, if an ART strip from VMM 5 is reported as strip number 30, the MM TP converts this to a global strip number of 350 ($5*64 + 30$). The units of the MM TP are strip pitches, and there is no conversion to a unit of meters. The MM TP then adds the distance from the beamline to the base of the MM chamber, in units of strips, to the global strip number, and divides the global strip number by the $z$-position of the chamber, to convert the strip to a slope, $x_\text{strip} / z_\text{strip}$.

\subsection{Adjustments for cosmic muons}
\label{sec:alg-crts}

The use of \textit{slopes} instead of \textit{strips} is motivated by the knowledge that particles arriving at the NSW from a proton-proton collision should travel in a nearly straight line. Thus, each layer of MM will record a different strip address for the incident particle, but a nearly identical slope. This simplifies logic for defining spatial \textit{roads} in the MM chamber.

However, for a cosmic ray test stand, the exact origin of incident particles are unknown. The MM TP algorithm is then modified to use strip addresses instead of slopes for the defintion of roads, as discussed in Section~\ref{sec:alg-finder}. The implementation of strip-to-slope conversion in the FPGA is nonetheless tested by multiplying each strip address by 1, instead of $1/z_\text{strip}$, to calculate a ``slope'' which is suitable for spatial coincidences of cosmic muons.

Additional adjustments to the MM TP algorithm for use at the cosmic ray test stand include: 

\subsection{Finder}
\label{sec:alg-finder}

Once the ART data is received and decoded, the next step of the MM TP algorithm is to filter the data into subsets which roughly look like an incident particle traversing the octuplet. This step is called the \textit{finder}.

The finder relies on heavy parallelization. A configurable number of roads are created in the FPGA to span the octuplet, and each roads evaluates in parallel whether is contains a sufficient number of ART strips to form a trigger. In this note, the road size is 1 road per VMM. This is demonstrated graphically in Figure~\ref{}.

Additionally, the roads should overlap to avoid edges cases where the incident muon travels very near to the boundary of two roads. This is implemented in the finder by allowing each road to use hits not only in their respective road, but also in neighboring roads. The number of neighboring roads is a configurable parameter; in this note, hits from one neighbor above and one neighbor below are used in a given road. For example, road 3 uses hits from VMM 2, VMM 3, and VMM 4, so that the effective road size is 3 VMMs.

The effective road size is chosen to maximize the acceptance of cosmic ray muons in the octuplet. 

A road must contain at least two hits on horizontal ($X$) planes and at least two hits on stereo ($UV$) planes to be satisfy a trigger. Additionally, at least two of the hits on $X$-planes must occur on opposite quadruplets, to ensure a good lever arm when fitting the $X$-hits to a line. These requirements are looser than the expected coincidence requirements at the NSW ($3X$ AND $3UV$), since the rate of noise and background hits is much lower for cosmic data-taking than at the LHC.

The finder also requires hits for a trigger to be within a time window in units of the BC clock. The expected time window for the NSW is two BCs; in this note, the window is increased to seven BCs to maximize efficiency and measure the efficiency of the window as a function of the size of the window. The time window is implemented as follows. When a hit arrives at the finder, it is stored in memory for seven BCs. For each road satisfied by the hit, no new hits are allowed in that road for that board until the hit is ``old'', i.e., after seven BCs. The stipulation that a hit in a road on a board cannot be overwritten by a new hit is motivated by the behavior of the detector and the ART data-flow: an incident muon is expected to leave a signal on multiple strips on a board (making a ``cluster''), each strip of which can send an ART signal typically separated by a couple of BCs. Only the first hit from the cluster should be used for the trigger, hence the earlier hit should not be overwritten by subsequent nearby hits. Then, as hits arrive, the coincidence requirements are evaluated for all roads in parallel. Once the coincidence requirements for a road are met and the earliest hit in the road becomes ``old'', a trigger is formed. Therefore hits are collected in a time window of seven BCs, and triggers cannot be formed until the oldest hit in the road is sufficiently old.

Every road is capable of forming one trigger per cycle of the BC clock. After evaluating all roads in parallel, the triggers created are placed in a priority encoder to be sent sequentially to the fitter for further calculations. The priority encoder sorts the triggers by road number, with smallest roads treated first. Because the MM TP clock is eight times faster than the BC clock, there can be at most eight triggers per BC.

\subsection{Fitter}
\label{sec:alg-fitter}

The triggers found by the finder are then sent to the \textit{fitter}, which calculates quantities of interest for the trigger. The fitter does not create or remove any triggers. The NSW implementation of the fitter is described first, and modifications for cosmic data-taking are described second.

For the NSW, the main deliverables of the fitter are the location and quality of a trigger. The location is represented by a region of interest (ROI), which is a number provided by downstream clients like Sector Logic mapping to $\eta-\phi$ space. The quality of the trigger is represented by the difference in angle a \textit{local} fit of the trigger ART hits and a \textit{global} fit. A \textit{local} fit is performed on only the ART hits, and a \textit{global} fit is performed with the constraint that the trigger be consistent with the interaction point.

The location of the trigger is derived in a Cartesian $m_x-m_y$ space, where $\hat{y}$ points perpendicular to the direction of the $X$-strips, $\hat{x}$ points parallel to the $X$-strips, and $m_i$ is the slope $i/z$\footnote{We apologize for the confusing notation. To recap: $X$, $U$, and $V$ denote different layers of micromegas. $x$, $y$, and $z$ are cartesian coordinates. The $X$-layers measure $m_y$.}. $m_y$ is measured as the average of slopes measured in each $X$-plane, $\overline{m_X}$. $m_x$ is measured as the difference between slopes measured in the $U$ and $V$ planes, with a geometric factor from the angle of the stereo strips: $\frac{m_U - m_V}{2\ \text{tan}(\theta_\text{stereo})}$.

\subsection{Output}
\label{sec:alg-output}

Finally, the MM TP algorithm stores copies of the input ART data and output trigger data in FIFO buffers, which are transmitted as UDP packets via ethernet to a computer for analysis.

\subsection{Bugs encountered}
\label{sec:alg-bugs}

By January 2017, the MM TP algorithm had been implemented in firmware and successfully created triggers in behavioral simulation. By summer 2017, the MM TP algorithm is implemented on a FPGA and records hundreds of thousands of cosmic muon triggers per day. Many bugs were uncovered along the way.



\section{Data taking}
\label{sec:data-taking}

The data described in this note were collected from May 2017 until July 2017. Each continuous period of data-taking is
 referred to as a ``run'', and the runs are listed in Table~\ref{tab:runs}. 
 One notes that here are many more MMTP events than MMFE8 events. There are several  reasons
 for this. The MMTP sends data whenever the finder finds a trigger, whereas the MMFE8 only sends
 data only when it receives a trigger signal from the scintillator (the scintillator trigger rate is about 1 Hz rate,
 and 60\% of the triggers contain a good track with 4 or more clusters and triggerable).

 For a good track, the MMTP  typically produces a handful of triggers 
 due to the overlapping roads. For example, a downward-going muon passing entirely through road 2
 also generates triggers in roads 1 and 3. In addition, if a track has more hits than the minimum required by the finder,
 when some of these hits get old additional triggers can be produced.
 A large number of triggers can be also produce by non-random noise.
 This seem to be the case for  run 3522 and 3530.
 This started a long investigation of the noise source. The cause has been now identified as
 EMI produced by the  the ART clock transmitted from the MMFE8 to the ADDC through an unshielded pairs of wires
 in the miniSAS cable and by the AC/DC converter powering the ADDC. 

 Between Runs 3522 and 3527, two front-end cards were repaired and replaced on the detector. Therefore, only
 data from Runs 3527, 3528, and 3530 are used to compare
the MMTP performance using different peaktimes in Sec.~\ref{sec:perf-integ}
%%%%%%%%%%%%%%%%%%%%%%%%%%%%%%%%
\begin{table}[!htpb]
\begin{center}
  \begin{tabular}{c | c | c | c | c}
    Run  & Dates         & MMFE8 Events & MMTP Events & Notes \\
    \hline
    3522 & 05/11 - 05/18 & 296532       & 15774567    & 200 ns integration time \\
    3527 & 06/26 - 07/02 & 233645       & 3635093     & 100 ns integration time \\
    3528 & 07/02 - 07/07 & 205053       & 3253330     & 50 ns integration time \\
    3530 & 07/19 - 07/24 & 192114       & 4060891     & 200 ns integration time \\
  \end{tabular}
  \caption{Run information.  }
\label{tab:runs}
\end{center}
\end{table}
%%%%%%%%%%%%%%%%%%%%%%%%%%%%%%%%%%%%%%%%%%%%%%%%%%%%

\subsection{Combining data streams}
\label{sec:data-streams}
As already mentioned, five data streams are produced and recorded on disk. In addition  to the scintillator and MMFE8 data streams,
the MMTP algorithm writes a data stream (TPFIT) with the trigger properties - including its BCID  as explained earlier,
 a data stream (GBT) with ADDC data in the 15 the BC preceding the
MMTP 
trigger, and a data stream (SCTBC) containing  the the scintillator trigger time converted into a 16-bit  BCID. 

Scintillator data, MMFE8 data, and  SCTBC data
 are easily combined (SC+MM data) using the event number and occasionally the DAQ time stamp.
Because the MMTP rate is much higher, we combine it with  SC+MM data through several steps: a) we select SC+MM data which contain MM clusters
 compatible with a track
and which also satisfy the MMTP finder; d) we select TPFIT events first with a DAQ time stamp within 100 ms from that of the SC+MM event and then
in a 3 BC window around SCTBC; c) of the remaining events in the TPFIT data, we pick the one which has more VMMs and boards in common with the MM clusters.

To associate GBT events to TPFIT events, we use the event number and BCID comparisons between BCID  trigger and ADDC hits.



\section{Performance}
\label{sec:perf}

The MMTP performs admirably.

\subsection{Low-level performance}
\label{sec:perf-simple}

\begin{figure}[!htpb]
  \begin{center}
    \includegraphics[width=0.4\textwidth]{figures/gbtanalysis3522/tpeff.pdf}
    \includegraphics[width=0.4\textwidth]{figures/tuna_analysis/trigger_hits_vs_event.pdf}
  \end{center}
  \vspace{-10pt}
  \caption{The efficiency of a trigger to be matched to a full readout event versus time (left) and the number of hits in the trigger as a function of time (right). Both figures show stable data-taking.}
  \label{fig:lowlevel}
\end{figure}

\begin{figure}[!htpb]
  \begin{center}
    \includegraphics[width=0.4\textwidth]{figures/tuna_analysis/trigger_nart.pdf}
    \includegraphics[width=0.4\textwidth]{figures/gbtanalysis3522/ang.pdf}
  \end{center}
  \vspace{-10pt}
  \caption{The number of hits in the MM TP and MM FE tracks (left) and the angle of the tracks (right). The track found by the trigger closely resembles the track found by the full readout, though with slightly fewer hits on average.}
  \label{fig:tp_vs_fe}
\end{figure}

\subsection{Roads}
\label{sec:perf-roads}

\subsection{Spatial and angular resolution}
\label{sec:perf-res}

\begin{figure}[!htpb]
  \begin{center}
    \includegraphics[width=0.4\textwidth]{figures/gbtanalysis3522/TP_xres_full.pdf}
    \includegraphics[width=0.4\textwidth]{figures/gbtanalysis3522/TP_xres.pdf}
  \end{center}
  \vspace{-10pt}
  \caption{The $x$ resolution of the MM TP relative to the full readout, using 1-VMM online roads (left) and 1/4-VMM offline roads (right). The tails of the resolution are greatly suppressed with smaller roads.}
  \label{fig:xres}
\end{figure}

\begin{figure}[!htpb]
  \begin{center}
    \includegraphics[width=0.4\textwidth]{figures/gbtanalysis3522/TP_yres_1road.pdf}
    \includegraphics[width=0.4\textwidth]{figures/gbtanalysis3522/TP_yres_smallroad.pdf}
  \end{center}
  \vspace{-10pt}
  \caption{The $y$ resolution of the MM TP relative to the full readout, using 1-VMM online roads (left) and 1/4-VMM offline roads (right). The tails of the resolution are greatly suppressed with smaller roads.}
  \label{fig:yres}
\end{figure}

\begin{figure}[!htpb]
  \begin{center}
    \includegraphics[width=0.4\textwidth]{figures/gbtanalysis3522/TP_angres_full.pdf}
    \includegraphics[width=0.4\textwidth]{figures/gbtanalysis3522/TP_angres.pdf}
  \end{center}
  \vspace{-10pt}
  \caption{The $\theta$ resolution of the MM TP relative to the full readout, using 1-VMM online roads (left) and 1/4-VMM offline roads (right). The tails of the resolution are greatly suppressed with smaller roads.}
  \label{fig:thetares}
\end{figure}

\begin{figure}[!htpb]
  \begin{center}
    \includegraphics[width=0.4\textwidth]{figures/gbtanalysis3522/TP_xres_angres_full.pdf}
    \includegraphics[width=0.4\textwidth]{figures/gbtanalysis3522/TP_xres_angres.pdf}
  \end{center}
  \vspace{-10pt}
  \caption{The $x$ and $\theta$ resolution of the MM TP relative to the full readout, using 1-VMM online roads (left) and 1/4-VMM offline roads (right). The tails of the resolution are greatly suppressed with smaller roads.}
  \label{fig:xthetares}
\end{figure}


\subsection{Time resolution}

\begin{figure}[!htpb]
  \begin{center}
    \includegraphics[width=0.4\textwidth]{figures/gbtanalysis3522/artwin_lin.pdf}
    \includegraphics[width=0.4\textwidth]{figures/gbtanalysis3522/artrpairs_lin.pdf}
  \end{center}
  \vspace{-10pt}
  \caption{The time window required to record all hits in a trigger (left) and the $\Delta\text{BC}$ of all pairs of hits in a trigger (right). A gaussian fit is overlaid on the distribution of $\Delta\text{BC}$ and describes the data well.}
  \label{fig:time}
\end{figure}

\begin{figure}[!htpb]
  \begin{center}
    \includegraphics[width=0.4\textwidth]{figures/gbtanalysis3522/avg_BCID.pdf}
    \includegraphics[width=0.4\textwidth]{figures/gbtanalysis3522/earliest_BCID.pdf}
  \end{center}
  \vspace{-10pt}
  \caption{The time resolution of the MM TP relative to the scintillator. The BCID of the trigger can be defined as the average BCID of the ART hits (left) or the earliest BCID (right). Choosing the average BCID has better resolution than choosing the earliest.}
  \label{fig:timeres}
\end{figure}

\subsection{Integration time}
\label{sec:perf-integ}

\begin{figure}[!htpb]
  \begin{center}
    \includegraphics[width=0.3\textwidth]{figures/gbtanalysis3530/artwin_lin.pdf}
    \includegraphics[width=0.3\textwidth]{figures/gbtanalysis3527/artwin_lin.pdf}
    \includegraphics[width=0.3\textwidth]{figures/gbtanalysis3528/artwin_lin.pdf}
  \end{center}
  \vspace{-10pt}
  \caption{The time window required to record all hits in a trigger for data collected with 200 ns (left), 100 ns (middle), and 50 ns (right) integration time in the VMM. The window decreases as the integration time decreases.}
  \label{fig:integ_window}
\end{figure}

\begin{figure}[!htpb]
  \begin{center}
    \includegraphics[width=0.3\textwidth]{figures/gbtanalysis3530/artrpairs_lin.pdf}
    \includegraphics[width=0.3\textwidth]{figures/gbtanalysis3527/artrpairs_lin.pdf}
    \includegraphics[width=0.3\textwidth]{figures/gbtanalysis3528/artrpairs_lin.pdf}
  \end{center}
  \vspace{-10pt}
  \caption{The $\Delta\text{BC}$ of all pairs of hits in a trigger for data collected with 200 ns (left), 100 ns (middle), and 50 ns (right) integration time in the VMM. The distribution is narrower as the integration time decreases.}
  \label{fig:integ_pairs}
\end{figure}

\begin{figure}[!htpb]
  \begin{center}
    \includegraphics[width=0.3\textwidth]{figures/gbtanalysis3530/avg_BCID.pdf}
    \includegraphics[width=0.3\textwidth]{figures/gbtanalysis3527/avg_BCID.pdf}
    \includegraphics[width=0.3\textwidth]{figures/gbtanalysis3528/avg_BCID.pdf}
  \end{center}
  \vspace{-10pt}
  \caption{The time resolution of the MM TP relative to the scintillator, where the BC of the trigger is defined as the average BCID of the ART hits, for data collected with 200 ns (left), 100 ns (middle), and 50 ns (right) integration time in the VMM.}
  \label{fig:integ_avg_bc}
\end{figure}

\begin{figure}[!htpb]
  \begin{center}
    \includegraphics[width=0.3\textwidth]{figures/gbtanalysis3530/earliest_BCID.pdf}
    \includegraphics[width=0.3\textwidth]{figures/gbtanalysis3527/earliest_BCID.pdf}
    \includegraphics[width=0.3\textwidth]{figures/gbtanalysis3528/earliest_BCID.pdf}
  \end{center}
  \vspace{-10pt}
  \caption{The time resolution of the MM TP relative to the scintillator, where the BC of the trigger is defined as the earliest BCID of the ART hits, for data collected with 200 ns (left), 100 ns (middle), and 50 ns (right) integration time in the VMM.}
  \label{fig:integ_avg_earliest}
\end{figure}



\section{Conclusion}
\label{sec:conclusions}

We investigated the response of the micromegas Address in Real Time (ART) and trigger processor.




\clearpage

\begin{thebibliography}{99}
\label{bibliography}
\setlength{\itemsep}{1.5pt plus 2.0pt minus 1.4pt}
\setlength{\parsep}{0pt}
\setlength{\parskip}{0pt}
\vspace{-6pt}

\bibitem{brian} B.~Clark et. al. An Algorithm for Micromegas Segment Reconstruction in the Level-1 Trigger of the New Small Wheel. \href{https://cds.cern.ch/record/1706160}{\color{blue}\underline{ATL-COM-UPGRADE-2014-012}}.
\bibitem{steve} S.~Chan et. al. Micromegas Trigger Processor Algorithm Performance in Nominal, Misaligned, and Misalignment Corrected Conditions. \href{https://cds.cern.ch/record/2113121}{\color{blue}\underline{ATL-COM-UPGRADE-2015-033}}.
\bibitem{nswtdr} ATLAS New Small Wheel Technical Design Report. \href{http://cds.cern.ch/record/1552862}{\color{blue}\underline{ATLAS-TDR-020}}.

\end{thebibliography}










\end{document} 

